\documentclass[final]{beamer}
  \mode<presentation>
  {
% you can chose your theme here:
%  \usetheme{Aachen}
%  \usetheme{I6dv}
%  \usetheme{I6pd}
  \usetheme{I6pd2}
%  \usetheme{I6td}
%  \usetheme{Oldi6}
}
  %\usepackage{calc} 
  %\usepackage{times}
  \usepackage{sfmath}            % for sans serif math fonts; wget http://dtrx.de/od/tex/sfmath.sty
  \usepackage{amsmath,amsthm, amssymb, latexsym}
  %\boldmath

  \usepackage[english]{babel}
  \usepackage[latin1]{inputenc}
  \usepackage[orientation=landscape,size=custom,width=200,height=120,scale=4,debug]{beamerposter}
  \graphicspath{{figures/}}
  \title[Fancy Posters]{Making Really Fancy Posters with \LaTeX and $\mathcal{M}\alpha_{t}\mbox{h}$ Symbols}
  \author[Dreuw \& Deselaers]{Philippe Dreuw and Thomas Deselaers}
  \institute[RWTH Aachen University]{Human Language Technology and Pattern Recognition,RWTH Aachen University}
  \newcommand{\footlinetext}{Lehrstuhl f\"ur Informatik 6 - Computer Science Department - RWTH Aachen University - Aachen, Germany \par Mail: \texttt{<surname>@cs.rwth-aachen.de} \hfill WWW: \texttt{http://www-i6.informatik.rwth-aachen.de}\vskip1ex}
  \date{Jul. 31th, 2007}

  \begin{document}
  \begin{frame}{} 
    \vfill
    \begin{block}{Huge Math Fonts}
        \Huge %
        \begin{equation}
          \sum_{e=1}^E\exp(e) \ne E=mc^2
        \end{equation}        
    \end{block}
    \vfill
    \begin{block}{Large Pi Fonts}
      \Large %
      $\mathit{\Pi}$ $\mathsf{\Pi}$ $\mathtt{\Pi}$ $\mathcal{\Pi}$ $\Pi$
    \end{block}
    \vfill
  \end{frame}
\end{document}



%%%%%%%%%%%%%%%%%%%%%%%%%%%%%%%%%%%%%%%%%%%%%%%%%%%%%%%%%%%%%%%%%%%%%%%%%%%%%%%%%%%%%%%%%%%%%%%%%%%%
%%% Local Variables: 
%%% mode: latex
%%% TeX-PDF-mode: t
%%% End: